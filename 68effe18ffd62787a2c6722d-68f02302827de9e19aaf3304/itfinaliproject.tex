\chapter{Helix Core, Swarm, and Jenkins Integration \\
\small{\textit{-- Johan Jaramillo}}}
\index{Helix Core}
\index{Helix Swarm}
\index{Jenkins}
\index{Chapter::Helix Core and Jenkins Integration}
\label{Chapter::Helix Core and Jenkins Integration}

\section{Overview}
For this individual final project, I deployed and integrated Perforce Helix Core,
Helix Swarm, and Jenkins to demonstrate a complete DevOps workflow for managing and
automating LaTeX document builds. The goal of this setup was to showcase enterprise-style
source control, code review, and continuous integration using Perforce tooling instead
of traditional Git-based solutions.

The workflow includes hosting a Perforce server, submitting and reviewing LaTeX source
files using Helix Swarm, and configuring Jenkins to automatically fetch and process the
latest files from Perforce as part of a CI pipeline.

\section{Helix Core (Perforce Server) Deployment}

\subsection{Droplet Setup}
A dedicated DigitalOcean droplet was provisioned to host the Helix Core server.
This droplet runs a Linux-based operating system and was configured exclusively
for Perforce services to maintain isolation and clarity in the deployment architecture.

\subsection{Installing Helix Core}
The Perforce Helix Core server (\texttt{p4d}) was installed using Perforce’s official
distribution. Once installed, the server was initialized with a dedicated root directory
for versioned data storage.

\begin{minted}{bash}
p4d -r /p4/1/root -p 1666
\end{minted}

The server listens on the default Perforce port \texttt{1666} and was verified using
the \texttt{p4 info} command after installation.

\subsection{User and Security Configuration}
A dedicated Perforce user account (\texttt{perforce}) was used as the primary service
account for administrative tasks and CI integration. Password-based authentication
was enabled using Perforce’s security levels.

During setup, the security level was temporarily lowered to reset credentials and
then restored to enforce authentication:

\begin{minted}{bash}
p4 configure set security=0
p4 passwd perforce
p4 configure set security=3
\end{minted}

This ensured that both interactive users and automated systems (such as Jenkins)
could authenticate securely with the Perforce server.

\section{Depot Creation and Initial File Submission}

\subsection{Creating a Depot}
A new Perforce depot named \texttt{latex-depot} was created to store all LaTeX source
files associated with the project. This depot serves as the single source of truth
for version-controlled documentation.

\subsection{Submitting LaTeX Files}
The LaTeX project files were uploaded from a client workspace mapped to the depot.
All source files, images, bibliography files, and class files were included to ensure
the project could be built consistently from version control alone.

\begin{minted}{bash}
p4 add ...
p4 submit -d "Initial LaTeX project submission"
\end{minted}

After the initial submission, at least one additional change was made and submitted
to demonstrate active version control usage, satisfying the assignment requirements.

\section{Helix Swarm Deployment and Code Review}

\subsection{Swarm Installation}
Helix Swarm was deployed on a separate DigitalOcean droplet and configured to connect
to the Helix Core server. Swarm was linked to the Perforce instance by specifying the
server address, service user, and authentication credentials during setup.

\subsection{Code Review Workflow}
Once integrated, Swarm was used to create and manage code reviews for submitted LaTeX
files. Each changelist submitted to the depot could be reviewed directly through the
Swarm web interface.

The review process included:
\begin{itemize}
  \item Viewing diffs of modified \texttt{.tex} files
  \item Adding review comments
  \item Approving changes through Swarm
\end{itemize}

This demonstrated an enterprise-style peer review workflow similar to pull requests
in Git-based systems.

\section{Jenkins Deployment}

\subsection{Jenkins Installation}
Jenkins was deployed on a separate droplet alongside an existing Overleaf instance.
This droplet was responsible for running CI jobs and interacting with the Perforce
server remotely.

Jenkins was installed using the official package repository and verified through
its web-based management interface.

\subsection{Perforce Integration in Jenkins}
The Jenkins Perforce Plugin was installed to enable native interaction with Helix Core.
A Jenkins credential was created using the \texttt{perforce} user credentials, allowing
Jenkins jobs to authenticate securely with the Perforce server.

This setup allows Jenkins pipelines to:
\begin{itemize}
  \item Connect to the remote Perforce server
  \item Sync the latest changelists from the depot
  \item Use changelist numbers as part of build metadata
\end{itemize}

\section{Preparation for CI/CD Automation}
With Helix Core, Swarm, and Jenkins successfully deployed and integrated, the environment
was fully prepared for CI/CD automation. Jenkins was able to authenticate against the
Perforce server and retrieve LaTeX source files, enabling automated compilation and
versioned PDF generation in later stages of the pipeline.

The Jenkins pipeline configuration and LaTeX compilation process are discussed in
subsequent sections of this report.
