\chapter{Helix Core, Swarm, and Jenkins Integration \\
\small{\textit{-- Johan Jaramillo}}}
\index{Helix Core}
\index{Helix Swarm}
\index{Jenkins}
\index{Chapter::Helix Core and Jenkins Integration}
\label{Chapter::Helix Core and Jenkins Integration}

\section{Overview}
For this individual final project, I deployed and integrated Perforce Helix Core,
Helix Swarm, and Jenkins to demonstrate a complete DevOps workflow for managing and
automating LaTeX document builds. The goal of this setup was to showcase enterprise-style
source control, code review, and continuous integration using Perforce tooling instead
of traditional Git-based solutions.

The workflow includes hosting a Perforce server, submitting and reviewing LaTeX source
files using Helix Swarm, and configuring Jenkins to automatically fetch and process the
latest files from Perforce as part of a CI pipeline.

\section{Helix Core (Perforce Server) Deployment}

\subsection{Droplet Setup}
A dedicated DigitalOcean droplet was provisioned to host the Helix Core server.
This droplet runs a Linux-based operating system and was configured exclusively
for Perforce services to maintain isolation and clarity in the deployment architecture.

\subsection{Installing Helix Core}
The Perforce Helix Core server (\texttt{p4d}) was installed using Perforce’s official
distribution. Once installed, the server was initialized with a dedicated root directory
for versioned data storage.

\begin{minted}{bash}
p4d -r /p4/1/root -p 1666
\end{minted}

The server listens on the default Perforce port \texttt{1666} and was verified using
the \texttt{p4 info} command after installation.

\subsection{User and Security Configuration}
A dedicated Perforce user account (\texttt{perforce}) was used as the primary service
account for administrative tasks and CI integration. Password-based authentication
was enabled using Perforce’s security levels.

During setup, the security level was temporarily lowered to reset credentials and
then restored to enforce authentication:

\begin{minted}{bash}
p4 configure set security=0
p4 passwd perforce
p4 configure set security=3
\end{minted}

This ensured that both interactive users and automated systems (such as Jenkins)
could authenticate securely with the Perforce server.

\section{Depot Creation and Initial File Submission}

\subsection{Creating a Depot}
A new Perforce depot named \texttt{latex-depot} was created to store all LaTeX source
files associated with the project. This depot serves as the single source of truth
for version-controlled documentation.

\subsection{Submitting LaTeX Files}
The LaTeX project files were uploaded from a client workspace mapped to the depot.
All source files, images, bibliography files, and class files were included to ensure
the project could be built consistently from version control alone.

\begin{minted}{bash}
p4 add ...
p4 submit -d "Initial LaTeX project submission"
\end{minted}

After the initial submission, one additional change was made and submitted
to demonstrate active version control usage, satisfying the assignment requirements.

\section{Helix Swarm Deployment and Code Review}

\subsection{Swarm Installation}
Helix Swarm was deployed on the same DigitalOcean droplet and configured to connect
to the Helix Core server. Swarm was linked to the Perforce instance by specifying the
server address, service user, and authentication credentials during setup.

\subsection{Code Review Workflow}
Once integrated, Swarm was used to create and manage code reviews for submitted LaTeX
files. Each changelist submitted to the depot could be reviewed directly through the
Swarm web interface.
\begin{figure}[H]
    \centering
    \includegraphics[width=0.9\textwidth]{png/PERFORCE.png}
    \caption{Perforce Helix Swarm .}
    \label{fig:overleaf-port}
\end{figure}


The review process included:
\begin{itemize}
  \item Viewing diffs of modified \texttt{.tex} files
  \item Adding review comments
  \item Approving changes through Swarm
\end{itemize}

This demonstrated an enterprise-style peer review workflow similar to pull requests
in Git-based systems.

\section{Jenkins Deployment}

\subsection{Perforce Plugin and SCM Configuration}
To enable Jenkins to interact with Helix Core, the Perforce plugin was installed
from the Jenkins plugin manager. This plugin provides native support for Perforce
workspaces, authentication, and changelist-based builds.

Within the Jenkins job configuration, Perforce was selected as the Source Code
Management (SCM) provider. A dedicated Perforce credential was created in Jenkins
using the service account, allowing Jenkins to authenticate securely with the Helix
Core server.

The SCM configuration defined:
\begin{itemize}
  \item The Perforce server address and port
  \item The Jenkins Perforce service user
  \item A client workspace specification managed automatically by Jenkins
\end{itemize}

This setup allowed Jenkins to dynamically create and manage Perforce workspaces
for each build execution.

The pipeline uses Docker to provide a consistent LaTeX build environment
and is designed to ignore LaTeX warnings while failing only if the expected
PDF output is not generated.

\begin{minted}{groovy}
pipeline {
    agent any

    environment {
        PDF_NAME = "itManual.pdf"
    }

    stages {

        stage('Checkout SCM') {
            steps {
                checkout scm
            }
        }

        stage('Build LaTeX') {
            steps {
                sh '''
                set +e

                echo "Running LaTeX build inside Docker (warnings ignored)..."

                docker run --rm \
                  -v "$PWD":/data \
                  -w /data \
                  texlive/texlive:latest \
                  latexmk -f -pdf -shell-escape -interaction=nonstopmode itManual.tex

                if [ ! -f "${PDF_NAME}" ]; then
                    echo "PDF not generated — failing build"
                    exit 1
                fi

                echo "PDF generated successfully"
                exit 0
                '''
            }
        }

        stage('Verify PDF') {
            steps {
                sh '''
                if [ ! -f "${PDF_NAME}" ]; then
                    echo "PDF not found"
                    exit 1
                fi
                echo "PDF verification passed"
                '''
            }
        }

        stage('Archive PDF') {
            steps {
                archiveArtifacts artifacts: "${PDF_NAME}", fingerprint: true
            }
        }
    }

    post {
        success {
            echo "Build completed successfully"
        }
        failure {
            echo "Build failed, check LaTeX logs"
        }
    }
}
\end{minted}

This pipeline demonstrates how Jenkins can be integrated with Perforce
to automatically retrieve source files, execute builds in a controlled
environment, and archive build artifacts. By validating only the existence
of the final PDF, the pipeline avoids unnecessary failures caused by
non-critical LaTeX warnings while still enforcing correct build output.


\begin{figure}[H]
    \centering
    \includegraphics[width=0.9\textwidth]{png/JENKINS.png}
    \caption{Perforce Helix Swarm .}
    \label{fig:overleaf-port}
\end{figure}

\subsection{Demonstration Scope Adjustment}

During the Jenkins integration demonstration, a reduced version of the LaTeX
project was used for the automated build. The full document relies heavily on
the \texttt{minted} package, which requires additional configuration for
shell-escape permissions and Python dependencies inside containerized
environments.

To ensure a reliable and repeatable CI demonstration within the project
constraints, a smaller LaTeX file was selected that still exercised the full
Perforce Jenkins workflow, including source retrieval, automated compilation,
and artifact archiving. This approach allowed the continuous integration
pipeline to be demonstrated clearly without being impacted by environment-specific
package limitations.

The same Jenkins configuration can be applied to the complete LaTeX project
once the container environment is extended to fully support all required
packages.