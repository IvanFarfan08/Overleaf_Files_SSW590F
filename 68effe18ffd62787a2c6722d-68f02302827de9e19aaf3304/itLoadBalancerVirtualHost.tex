\chapter{Load Balancer and Virtual Hosts \\
\small{\textit{-- Ivan Farfan, Johan Jaramillo, Ryan Davis}}}
\index{Load Balancer and Virtual Hosts} 
\index{Chapter!LBVH}
\label{Chapter::LBVH}
\section*{Load Balancer}

\subsection*{1. Full Documentation}

This project sets up an Nginx reverse proxy to act as a load balancer between two simple web servers (\texttt{web1} and \texttt{web2}). The load balancer distributes incoming requests.

\begin{itemize}
    \item \textbf{Technology Used:} Docker, Nginx, Docker Compose
    \item \textbf{Directory:} \texttt{load-balanced-app/}
    \item \textbf{Key Files:}
    \begin{itemize}
        \item \texttt{docker-compose.yml} – defines services: \texttt{lb-web1}, \texttt{lb-web2}, \texttt{loadbalancer}
        \item \texttt{nginx/nginx.conf} – defines an upstream block and load balancing config
        \item \texttt{web1/index.html} and \texttt{web2/index.html} – backend HTML responses
    \end{itemize}
    \item \textbf{How It Works:} When visiting the load balancer URL and refreshing, the response alternates between \texttt{lb-web1} and \texttt{lb-web2}, proving the round-robin behavior.
\end{itemize}

\subsection*{2. Access Link}

\begin{itemize}
    \item \textbf{Load Balancer URL:} \url{http://143.198.5.82:8080/}
\end{itemize}




\section*{Virtual Hosting / Reverse Proxy}

\subsection*{1. Full Documentation}

This configuration sets up Nginx to use virtual hosting, routing traffic to different web services based on the \texttt{Host} header. Visiting different hostnames returns different content.

\begin{itemize}
    \item \textbf{Technology Used:} Docker, Nginx, Docker Compose
    \item \textbf{Directory:} \texttt{virtual-host-app/}
    \item \textbf{Key Files:}
    \begin{itemize}
        \item \texttt{docker-compose.yml} – defines services: \texttt{web1}, \texttt{web2}, \texttt{proxy}
        \item \texttt{nginx/nginx.conf} – contains server blocks for \texttt{web1.domain.com} and \texttt{web2.domain.com}
        \item \texttt{web1/index.html} and \texttt{web2/index.html} – separate virtual host content
    \end{itemize}
    \item \textbf{How It Works:} Nginx listens on port 80 and checks the hostname. Requests to \texttt{web1.domain.com} are proxied to the \texttt{web1} container; \texttt{web2.domain.com} goes to \texttt{web2}.
\end{itemize}

\subsection*{2. Access Instructions}

To access the virtual host links below, you must add the following lines to your local machine’s \texttt{/etc/hosts} file:

\begin{verbatim}
143.198.5.82 web1.domain.com
143.198.5.82 web2.domain.com
\end{verbatim}

\subsection*{3. Access Links}

\begin{itemize}
    \item \textbf{Web1:} \url{http://web1.domain.com/}
    \item \textbf{Web2:} \url{http://web2.domain.com/}
\end{itemize}

\subsection*{4. Screenshots}

\begin{figure}[h]
    \centering
    \includegraphics[width=0.6\textwidth]{png/LBVH/web1Screenshot.png}
    \caption{Load balancer web 1 page.}
\end{figure}

\begin{figure}[h]
    \centering
    \includegraphics[width=0.6\textwidth]{png/LBVH/web2Screenshot.png}
    \caption{Load balancer web 2 page.}
\end{figure}

\begin{figure}[h]
    \centering
    \includegraphics[width=0.6\textwidth]{png/LBVH/virtualWeb1.png}
    \caption{Virtually hosted web1.domain.com}
\end{figure}

\begin{figure}[h]
    \centering
    \includegraphics[width=0.6\textwidth]{png/LBVH/virtualWeb2.png}
    \caption{Virtually hosted web2.domain.com}
\end{figure}