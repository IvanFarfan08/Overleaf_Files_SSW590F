\chapter{Overleaf and Github Integration \\
\small{\textit{-- Ivan Farfan, Johan Jaramillo, Ryan Davis}}}
\index{Overleaf and Github Integration} 
\index{Chapter::Overleaf and Github Integration}
\label{Chapter::Overleaf and Github Integration}

\section{Overview}
For this assignment, we performed the following tasks: getting a domain via the GitHub Student Developer Pack, getting an SSL certificate for our overleaf domain, configure Latex / Overleaf to support all packages, connecting Overleaf to Github for project version control, compiling TEX projects from the command line.

\section{Domain Name}
We decided to register the domain \texttt{ssw590f25.me} using the GitHub Student Developer Pack, which provided a free one-year domain through Namecheap. This allowed us to set up a professional web address for our project with minimal cost and effort. We had to add an A record using the Overleaf \hyperref[Chapter::Hosts]{container droplet IP} as host so that all traffic to the domain gets redirected to the local version of overleaf we're hosting.

\section{SSL Configuration}

To enable HTTPS on our Overleaf container, we installed and configured SSL certificates from Let’s Encrypt using \texttt{certbot}. This ensures secure encrypted communication between users and the server.

\begin{minted}{bash}
apt install python3-certbot-nginx nginx
sudo certbot --nginx certonly
\end{minted}

Once the certificate was obtained, we configured Nginx to serve as a reverse proxy — forwarding client requests to the Overleaf service while handling SSL termination.

\begin{minted}{nginx}
server {
    listen 80;
    server_name overleaf.ssw590f25.me;
    return 301 https://$server_name$request_uri;
}

server {
    listen 443 ssl;
    server_name overleaf.ssw590f25.me;

    ssl_certificate /etc/letsencrypt/live/overleaf.ssw590f25.me/fullchain.pem;
    ssl_certificate_key /etc/letsencrypt/live/overleaf.ssw590f25.me/privkey.pem;

    location / {
        proxy_pass http://localhost:8877;
        proxy_set_header X-Forwarded-For $remote_addr;
    }
}
\end{minted}

Configuring Nginx in this way is crucial because it acts as a reverse proxy—managing secure HTTPS traffic, terminating SSL connections, and forwarding requests to the internal Overleaf container. 

\section{Configuring LaTeX Packages in Overleaf (Container)}

To enable full LaTeX package support inside the Overleaf container, we used the following commands:

\subsection*{Steps}

\paragraph{1) Update \texttt{tlmgr} (TeX Live Manager).}
\begin{minted}{bash}
docker exec sharelatex tlmgr update --self
\end{minted}
\emph{Why:} Brings the package manager itself up-to-date so subsequent installs use the latest metadata.

\paragraph{2) Install the full TeX Live package set.}
\begin{minted}{bash}
docker exec sharelatex tlmgr install scheme-full
\end{minted}
\emph{Why:} \texttt{scheme-full} pulls essentially all TeX Live packages, covering most class/style/font dependencies you’ll encounter.
\textbf{Note:} This took a long time and it's not permanent, it will reset to the basic Latex version if the container is rebuilt.

\section{Connecting Overleaf to GitHub}

We connected our Overleaf instance at \texttt{overleaf.ssw590f25.me} to GitHub to automatically sync compiled project files.

\subsection*{Generating SSH Keys and Configuring Access}

\begin{minted}{bash}
ssh-keygen -t ed25519 -f /root/.ssh/id_ed25519_exports -N "" -C "overleaf-exports"
ssh-keyscan github.com >> /root/.ssh/known_hosts
chmod 600 /root/.ssh/known_hosts
cat /root/.ssh/id_ed25519_exports.pub
ssh -i /root/.ssh/id_ed25519_exports -T git@github.com
\end{minted}

Response:
\begin{verbatim}
Hi IvanFarfan08/Overleaf_Files_SSW590F! You've successfully authenticated, 
but GitHub does not provide shell access.
\end{verbatim}

Configured Git to always use this key:
\begin{minted}{bash}
cat >> /root/.ssh/config <<'EOF'
Host github.com
  User git
  HostName github.com
  IdentityFile ~/.ssh/id_ed25519_exports
  IdentitiesOnly yes
EOF
chmod 600 /root/.ssh/config
\end{minted}

The public key was then added as a deployment key in the GitHub repository.

\subsection*{Creating Export Service}

\begin{minted}{bash}
SRC="/root/Overleaf_container/toolkit/data/overleaf/data/compiles"
DST="/root/overleaf_exports"
REMOTE="git@github.com:IvanFarfan08/Overleaf_Files_SSW590F.git"

sudo apt update
sudo apt install -y git inotify-tools rsync
\end{minted}

Initialized the staging repository:

\begin{minted}{bash}
sudo mkdir -p "$DST"
cd "$DST"
git init -b main
git config user.name  "Overleaf Export Bot"
git config user.email "export@overleaf.local"

cat > .gitignore <<'EOF'
*.aux
*.log
*.out
*.synctex.gz
EOF

git add -A
git commit -m "Initial staging repo"

ssh-keyscan github.com >> /root/.ssh/known_hosts
chmod 600 /root/.ssh/known_hosts
GIT_SSH_COMMAND="ssh -i $KEY -o StrictHostKeyChecking=yes" git push -u origin main
\end{minted}

\subsection*{Watcher Script}

\begin{minted}{bash}
sudo tee /usr/local/bin/compiles-export-push.sh >/dev/null <<'EOF'
#!/usr/bin/env bash
set -euo pipefail

SRC="/root/Overleaf_container/toolkit/data/overleaf/data/compiles"
DST="/root/overleaf_exports"
BRANCH="main"
KEY="/root/.ssh/id_ed25519_exports"

export GIT_SSH_COMMAND="ssh -i $KEY -o StrictHostKeyChecking=yes"

mkdir -p "$DST"

copy_and_commit() {
  local top="$1"
  rsync -a --delete "$SRC/$top/" "$DST/$top/"
  cd "$DST"
  git add "$top"
  if ! git diff --cached --quiet; then
    msg="export: $top @ $(date -u +'%Y-%m-%d %H:%M:%S %Z')"
    git commit -m "$msg"
    git push origin "$BRANCH"
    echo "Pushed: $msg"
  fi
}

DEBOUNCE=5
declare -A last_at

inotifywait -m -r \
  -e close_write,modify,attrib,create,delete,move \
  --format '%w%f' "$SRC" | while read -r path; do
    rel="${path#"$SRC/"}"
    top="${rel%%/*}"
    [ -n "$top" ] || continue

    now=$(date +%s)
    last=${last_at[$top]:-0}
    if (( now - last >= DEBOUNCE )); then
      last_at[$top]=$now
      echo "Detected change in $top"
      copy_and_commit "$top" || echo "Commit/push failed for $top"
    fi
  done
EOF

sudo chmod +x /usr/local/bin/compiles-export-push.sh
\end{minted}

This script continuously monitors the Overleaf compile directory for any changes using
\texttt{inotifywait}. Whenever a project is recompiled, it automatically syncs the updated
subfolder to the local Git staging directory using \texttt{rsync}, commits the changes, and
pushes them to the GitHub repository.  
The \texttt{DEBOUNCE} variable ensures that each subfolder is only pushed once every few
seconds, preventing redundant commits during rapid file writes.

\subsection*{Systemd Service}

\begin{minted}{bash}
sudo tee /etc/systemd/system/compiles-export-push.service >/dev/null <<'EOF'
[Unit]
Description=Export changed Overleaf compiles subfolders to root-owned repo and push
After=network-online.target
Wants=network-online.target

[Service]
Type=simple
User=root
ExecStart=/usr/local/bin/compiles-export-push.sh
Restart=always
RestartSec=2

[Install]
WantedBy=multi-user.target
EOF

sudo systemctl daemon-reload
sudo systemctl enable --now compiles-export-push.service
\end{minted}

The Systemd service runs the watcher script automatically in the background.
It starts on boot, restarts on failure, and ensures continuous syncing between
Overleaf and GitHub. By enabling it with \texttt{systemctl enable --now}, the
service remains active and constantly monitors the compile directory without
manual intervention. The service keeps updating projects (including this one) and is available at \href{https://github.com/IvanFarfan08/Overleaf_Files_SSW590F}{this Github repo}.

\section{Compiling Overleaf Projects from the Command Line}

We found that compiled versions of Overleaf projects are stored in 
toolkit/data/overleaf/data/compiles. 

Each folder in this directory is mapped to a unique \texttt{projectID\_userID} pair.

This means that one can enter the Docker image created in 
\hyperref[Chapter::LaTeXDocker]{Chapter~\ref*{Chapter::LaTeXDocker}} and use the instructions there to compile the project files.  
However, Overleaf automatically generates compiled PDF files for each project, 
and these can already be found in the corresponding folder under the 
\texttt{compiles} directory.