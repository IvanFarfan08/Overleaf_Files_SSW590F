\chapter{Overleaf and Github Integration \\
\small{\textit{-- Ivan Farfan, Johan Jaramillo, Ryan Davis}}}
\index{Overleaf and Github Integration} 
\index{Chapter::Overleaf and Github Integration}
\label{Chapter::Overleaf and Github Integration}

\section{Overview}
For this assignment, we performed the following tasks: getting a domain via the GitHub Student Developer Pack, getting an SSL certificate for our overleaf domain, configure Latex / Overleaf to support all packages, connecting Overleaf to Github for project version control, compiling TEX projects from the command line.

\section{Domain Name}
We decided to register the domain \texttt{ssw590f25.me} using the GitHub Student Developer Pack, which provided a free one-year domain through Namecheap. This allowed us to set up a professional web address for our project with minimal cost and effort. We had to add an A record using the Overleaf \hyperref[Chapter::Hosts]{container droplet IP} as host so that all traffic to the domain gets redirected to the local version of overleaf we're hosting.

\section{SSL Configuration}

To enable HTTPS on our Overleaf container, we installed and configured SSL certificates from Let’s Encrypt using \texttt{certbot}. This ensures secure encrypted communication between users and the server.

\begin{minted}{bash}
apt install python3-certbot-nginx nginx
sudo certbot --nginx certonly
\end{minted}

Once the certificate was obtained, we configured Nginx to serve as a reverse proxy — forwarding client requests to the Overleaf service while handling SSL termination.

\begin{minted}{nginx}
server {
    listen 80;
    server_name overleaf.ssw590f25.me;
    return 301 https://$server_name$request_uri;
}

server {
    listen 443 ssl;
    server_name overleaf.ssw590f25.me;

    ssl_certificate /etc/letsencrypt/live/overleaf.ssw590f25.me/fullchain.pem;
    ssl_certificate_key /etc/letsencrypt/live/overleaf.ssw590f25.me/privkey.pem;

    location / {
        proxy_pass http://localhost:8877;
        proxy_set_header X-Forwarded-For $remote_addr;
    }
}
\end{minted}

Configuring Nginx in this way is crucial because it acts as a reverse proxy—managing secure HTTPS traffic, terminating SSL connections, and forwarding requests to the internal Overleaf container. 

\section{Configuring LaTeX Packages in Overleaf (Container)}

To ensure our Overleaf container can compile \emph{any} project (including those requiring uncommon packages), we installed the full TeX Live scheme inside the running container.

\subsection*{Steps}

\paragraph{1) Update \texttt{tlmgr} (TeX Live Manager).}
\begin{minted}{bash}
# Replace 'sharelatex' with your container name if different
docker exec sharelatex tlmgr update --self
\end{minted}
\emph{Why:} Brings the package manager itself up-to-date so subsequent installs use the latest metadata.

\paragraph{2) Install the full TeX Live package set.}
\begin{minted}{bash}
docker exec sharelatex tlmgr install scheme-full
\end{minted}
\emph{Why:} \texttt{scheme-full} pulls essentially all TeX Live packages, covering most class/style/font dependencies you’ll encounter.
\textbf{Note:} This takes a long time and uses many GB of disk space.