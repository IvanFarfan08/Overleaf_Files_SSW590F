\chapter{Jenkins with Pytest \\
\small{\textit{-- Ivan Farfan, Johan Jaramillo, Ryan Davis}}}
\index{Jenkins CI/CD}
\index{Chapter::Jenkins with Pytest}
\label{Chapter::JenkinsPytest}

\section{Containerized Jenkins Setup}

\subsection*{Dockerfile with Python Support}

The default Jenkins image (\texttt{jenkins/jenkins:lts}) does not include Python, which caused the initial pipeline to fail when attempting to create a virtual environment.  
To fix this, a custom Dockerfile was created to install Python~3, \texttt{pip}, and \texttt{venv} directly into the Jenkins image.

\begin{minted}{docker}
FROM jenkins/jenkins:lts
USER root

RUN apt-get update && \
    apt-get install -y python3 python3-pip python3-venv && \
    rm -rf /var/lib/apt/lists/*

USER jenkins
\end{minted}

\subsection*{docker-compose.yml Configuration}

The Docker Compose file was modified to build the new image and ensure Jenkins runs as \texttt{root}, preventing permission issues when writing to the mounted volume.

\begin{minted}{yaml}
version: '3.8'
services:
  jenkins:
    build: .
    container_name: jenkins
    user: root
    ports:
      - "8080:8080"
      - "50000:50000"
    volumes:
      - jenkins_home:/var/jenkins_home
      - /var/run/docker.sock:/var/run/docker.sock
    environment:
      - JENKINS_OPTS=--prefix=/jenkins
volumes:
  jenkins_home:
\end{minted}

Jenkins was then rebuilt and started using:

\begin{minted}{bash}
docker compose down
docker compose build
docker compose up -d
docker exec -it jenkins python3 --version
\end{minted}

Once running, Jenkins was accessed through:
\begin{verbatim}
http://localhost:8080/jenkins
\end{verbatim}

\begin{center}
\includegraphics[width=0.8\textwidth]{png/jenkins_running.png}\\
\textit{Figure: Jenkins container running successfully in Docker.}
\end{center}





\section{Pipeline Configuration in Jenkins}

\subsection*{Final Jenkinsfile}

After multiple iterations, the working Jenkinsfile avoided using virtual environments and installed dependencies system-wide using the \texttt{--break-system-packages} flag.  
This resolved Python’s \texttt{externally-managed-environment} error, which had caused earlier builds to fail.

\begin{minted}{groovy}
pipeline {
    agent any
    stages {
        stage('Install dependencies') {
            steps {
                sh 'pip3 install --break-system-packages -r requirements.txt'
            }
        }
        stage('Run tests') {
            steps {
                sh 'pytest --junitxml=report.xml || true'
            }
        }
        stage('Publish Report') {
            steps {
                junit 'report.xml'
            }
        }
    }
}
\end{minted}

The \texttt{|| true} ensures the pipeline continues to publish the report even if some tests fail.

\subsection*{GitHub Integration}

The Jenkins job was configured as a \textit{Pipeline script from SCM} job with the following settings:

\begin{itemize}
    \item \textbf{Repository:} \texttt{https://github.com/Johan0214/jenkins-python-pytest-demo.git}
    \item \textbf{Branch:} \texttt{main}
    \item \textbf{Script Path:} \texttt{Jenkinsfile}
\end{itemize}



\section{Results}

\subsection*{Pipeline Execution}

After configuration, the pipeline successfully completed all stages:
\begin{itemize}
    \item Checkout SCM
    \item Install dependencies
    \item Run tests
    \item Publish report
\end{itemize}

Even though one test intentionally failed, the report was still generated and displayed in Jenkins.

\begin{center}
\includegraphics[width=0.8\textwidth]{png/pipeline_stages.png}\\
\textit{Figure: Jenkins pipeline execution showing all stages completed.}
\end{center}

\subsection*{Test Results}

Jenkins displayed the following results:
\begin{itemize}
    \item \textbf{2 tests passed}
    \item \textbf{1 test failed (intentional)}
\end{itemize}


\begin{center}
\includegraphics[width=0.8\textwidth]{png/test_results.png}\\
\textit{Figure: Jenkins test report showing 2 passed and 1 failed test.}
\end{center}

\subsection*{Error Handling and Fixes}

Several configuration problems were resolved during the process:
\begin{itemize}
    \item \textbf{Python missing:} resolved by building a custom Docker image.
    \item \textbf{Permission denied:} fixed by running Jenkins as \texttt{root}.
    \item \textbf{Virtual environment errors:} avoided by using direct \texttt{pip3 install}.
    \item \textbf{Test report not publishing:} solved using the \texttt{|| true} modifier.
\end{itemize}

\section{Conclusion}

The Jenkins–Pytest pipeline was successfully implemented using a custom Docker-based Jenkins environment.  


