\chapter{Linux Commands \\
\small{\textit{-- Ivan Farfan, Johan Jaramillo, Ryan Davis}}}
\index{Linux Commands} 
\index{Chapter!Linux Commands}
\label{chap:linux}

\section*{Environment Setup}
The following bash commands were executed in a Linux terminal to create the test environment:

\begin{verbatim}
mkdir -p ~/lx-test && cd ~/lx-test
printf "alpha\nbeta\nGamma\ngamma\nbeta\n" > words.txt
printf "id,name,dept\n1,Ada,EE\n2,Linus,CS\n3,Grace,EE\n4,Dennis,CS\n" > people.csv
printf "INFO boot ok\nWARN disk low\nERROR fan fail\nINFO shutdown\n" > sys.log
dd if=/dev/zero of=blob.bin bs=1K count=48 status=none
mkdir -p src/lib tmp archive
printf "one two three four\n" > src/file1.txt
printf "two three four five\n" > src/file2.txt
ln -s src/file1.txt link-to-file1
touch -t 202401020304 old.txt
\end{verbatim}


\section*{A) Navigation \& File Ops}
\begin{enumerate}
\item Show your present working directory path only. \vspace{2em}\\
\begin{verbatim}
    pwd
    /senior_year/ssw590-devops/hw/linuxCommands/lx-test

\end{verbatim}

\item List all entries in the current directory, one per line, including dotfiles. \vspace{2em}\\
\begin{verbatim}
ls -a1
    .
    ..
    archive
    blob.bin
    link-to-file1
    old.txt
    people.csv
    src
    sys.log
    tmp
    words.txt

\end{verbatim}

\item Copy src/file1.txt to tmp/ only if tmp exists; do it verbosely. \vspace{2em}\\
\begin{verbatim}
[ -d tmp ] && cp -v src/file1.txt tmp/
    src/file1.txt -> tmp/file1.txt
\end{verbatim}
\item Move old.txt into archive/ and keep its original timestamp. \vspace{2em}\\
\begin{verbatim}
cp -p old.txt archive/ && rm old.txt

\end{verbatim}
\item Create a new empty file notes.md only if it doesn’t already exist. \vspace{2em}\\
\begin{verbatim}
touch -c newfile.txt

\end{verbatim}
\item Show disk usage (human-readable) for the src directory only (not total FS). \vspace{2em}\\
\begin{verbatim}
    du -sh src

8.0K	src
\end{verbatim}
\end{enumerate}

\section*{B) Viewing \& Searching}
\begin{enumerate}
\setcounter{enumi}{6}
\item Print line numbers while displaying sys.log. \vspace{2em}\\
\begin{verbatim}
nl -ba sys.log

     1	INFO boot ok
     2	WARN disk low
     3	ERROR fan fail
     4	INFO shutdown

\end{verbatim}

\item Show only the lines in sys.log that contain ERROR (case-sensitive). \vspace{2em}\\
\begin{verbatim}
cat sys.log|grep ERROR
    ERROR fan fail
\end{verbatim}
\item Count how many distinct words appear in words.txt (case-insensitive). \vspace{2em}\\
\begin{verbatim}
tr -c '[:alnum:]' '[\n*]' < words.txt | tr '[:upper:]' '[:lower:]' | sort | uniq | wc -l

       3

\end{verbatim}
\item From words.txt, show lines that start with g or G. \vspace{2em}\\
\begin{verbatim}
grep -ci "^g" words.txt 
    2
\end{verbatim}
\item Display the first 2 lines of people.csv without using an editor. \vspace{2em}\\
\begin{verbatim}
head -n 2 people.csv
    id,name,dept
    1,Ada,EE
\end{verbatim}
\item Show the last 3 lines of sys.log and keep following if the file grows. \vspace{2em}\\
\begin{verbatim}
tail -n 3 -f sys.log
    WARN disk low
    ERROR fan fail
    INFO shutdown



\end{verbatim}
\end{enumerate}

\section*{C) Text Processing}
\begin{enumerate}
\setcounter{enumi}{12}
\item From people.csv, print only the name column (2nd), excluding the header. \vspace{2em}\\
\begin{verbatim}
tail -n +2 people.csv | cut -d, -f2

    Ada
    Linus
    Grace
    Dennis
\end{verbatim}

\item Sort words.txt case-insensitively and remove duplicates. \vspace{2em}\\
\begin{verbatim}
sort -fu words.txt

    alpha
    beta
    Gamma
\end{verbatim}
\item Replace every three with 3 in all files under src/ in-place, creating .bak backups. \vspace{2em}\\
\begin{verbatim}
ryandavis@Ryans-MacBook-Air lx-test % find src -type f -exec sed -i .bak -E 's/\<three\>/3/g' {} +
ryandavis@Ryans-MacBook-Air lx-test % ls
    archive		link-to-file1	src		tmp
    blob.bin	people.csv	sys.log		words.txt
ryandavis@Ryans-MacBook-Air lx-test % ls src 
    file1.txt	file1.txt.bak	file2.txt	file2.txt.bak	lib
    
\end{verbatim}
\item Print the number of lines, words, and bytes for every *.txt file in src/. \vspace{2em}\\
\begin{verbatim}
wc src/*.txt

       1       4      19 src/file1.txt
       1       4      20 src/file2.txt
       2       8      39 total

\end{verbatim}
\end{enumerate}

\section*{D) Permissions \& Ownership}
\begin{enumerate}
\setcounter{enumi}{16}
\item Make tmp/ readable, writable, and searchable only by the owner. \vspace{2em}\\
\begin{verbatim}
chmod 700 tmp/

\end{verbatim}
\item Give group execute permission to src/lib recursively without touching others/owner bits. \vspace{2em}\\
\begin{verbatim}
chmod -R g+x src/lib

\end{verbatim}
\item Show the numeric (octal) permissions of src/file2.txt. \vspace{2em}\\
\begin{verbatim}
stat -f "%Lp" src/file2.txt
    644
\end{verbatim}
\item Make notes.md append-only for the owner via file attributes (if supported). \vspace{2em}\\
\begin{verbatim}
    not supported
\end{verbatim}
\end{enumerate}

\section*{E) Links \& Find}
\begin{enumerate}
\setcounter{enumi}{20}
\item Verify whether link-to-file1 is a symlink and show its target path. \vspace{2em}\\
\begin{verbatim}
ls -l link-to-file1
lrwxrwxrwx 1 root root 13 Sep 17 20:46 link-to-file1 -> src/file1.txt
\end{verbatim}
\item Find all regular files under the current tree larger than 40 KiB. \vspace{2em}\\
\begin{verbatim}
find . -type f -size +40k
./blob.bin
\end{verbatim}
\item Find files modified in the last 10 minutes under tmp/ and print their sizes. \vspace{2em}\\
\begin{verbatim}
 find tmp/ -type f -mmin -10 -exec ls -lh {} \;
-rw-r--r-- 1 root root 19 Sep 17 20:57 tmp/file1.txt
\end{verbatim}
\end{enumerate}

\section*{F) Processes \& Job Control}
\begin{enumerate}
\setcounter{enumi}{23}
\item Show your processes in a tree view. \vspace{2em}\\
\begin{verbatim}
ps --forest
  PID TTY          TIME CMD
    1 pts/0    00:00:00 bash
  827 pts/0    00:00:00 ps
\end{verbatim}
\item Start sleep 120 in the background and show its PID. \vspace{2em}\\
\begin{verbatim}
sleep 120 & echo $!
[1] 834
834
\end{verbatim}
\item Send a TERM signal to all sleep processes owned by you (don’t use kill -9). \vspace{2em}\\
\begin{verbatim}
pkill sleep
[1]+  Done                    sleep 120
\end{verbatim}
\item Show the top 5 processes by memory usage (one-shot, not interactive). \vspace{2em}\\
\end{enumerate}
\begin{verbatim}
ps aux --sort=-\%mem | head -n 6
USER       PID \%CPU \%MEM    VSZ   RSS TTY      STAT START   TIME COMMAND
root         1  0.0  0.0   4632  3808 pts/0    Ss   Sep17   0:00 bash
root       836  0.0  0.0   6408  2496 pts/0    R+   00:58   0:00 ps aux --sort=-\%mem
root       828  0.0  0.0   2304  1292 pts/1    Ss+  Sep17   0:00 /bin/sh
root       837  0.0  0.0   2212  1104 pts/0    S+   00:58   0:00 head -n 6
\end{verbatim}
\section*{G) Archiving \& Compression}
\begin{enumerate}
\setcounter{enumi}{27}
\item Create a gzipped tar archive src.tgz from src/ with relative paths. \vspace{2em}\\
\begin{verbatim}
tar -czf src.tgz src/
\end{verbatim}
\item List the contents of src.tgz without extracting. \vspace{2em}\\
\begin{verbatim}
tar -tzf src.tgz
src/
src/file1.txt
src/file1.txt.bak
src/file2.txt
src/file2.txt.bak
src/lib/
\end{verbatim}
\item Extract only file2.txt from src.tgz into tmp/. \vspace{2em}\\
\begin{verbatim}
tar -xzf src.tgz -C tmp/ src/file2.txt3
\end{verbatim}
\end{enumerate}

\section*{H) Networking \& System Info}
\begin{enumerate}
\setcounter{enumi}{30}
\item Show all listening TCP sockets with associated PIDs (no root assumptions). \vspace{2em}\\
\begin{verbatim}
ss -tlnp
State   Recv-Q   Send-Q     Local Address:Port     Peer Address:Port  Process
\end{verbatim}
\item Print your default route (gateway) in a concise form. \vspace{2em}\\
\begin{verbatim}
ip route | grep default
default via 172.17.0.1 dev eth0 
\end{verbatim}
\item Display kernel name, release, and machine architecture. \vspace{2em}\\
\begin{verbatim}
uname -srm
Linux 6.10.14-linuxkit aarch64
\end{verbatim}
\item Show the last 5 successful logins (or last sessions) on the system. \vspace{2em}\\
\begin{verbatim}
last -n 5

wtmp begins Wed Sep 17 20:44:48 2025
\end{verbatim}
\end{enumerate}

\section*{I) Package \& Services (Debian/Ubuntu)}
\begin{enumerate}
\setcounter{enumi}{34}
\item Show the installed version of package coreutils. \vspace{2em}\\
\begin{verbatim}
dpkg -l coreutils
\end{verbatim}
\item Search available packages whose names contain ripgrep. \vspace{2em}\\
\begin{verbatim}
apt search ripgrep
Sorting... Done
Full Text Search... Done
ripgrep/jammy-updates,jammy-security 13.0.0-2ubuntu0.1 arm64
  Recursively searches directories for a regex pattern
\end{verbatim}
\item Check whether service cron is active and print its status line only. 
\vspace{2em}\\
\begin{verbatim}
service cron status
\end{verbatim}
\end{enumerate}

\section*{J) Bash \& Scripting}
\begin{enumerate}
\setcounter{enumi}{37}
\item Write a one-liner that loops over *.txt in src/ and prints: \texttt{<filename>: <linecount>}. \vspace{2em}\\
\begin{verbatim}
for f in src/*.txt; do echo "$f: $(wc -l < "$f")"; done
\end{verbatim}
\item Write a command that exports CSV rows where dept == "CS" to cs.txt (exclude header). \vspace{2em}\\
\begin{verbatim}
awk -F, '$2 == "CS" {print}' input.csv | tail -n +2 > cs.txt
\end{verbatim}
\item Create a variable X with value 42, print it, then remove it from the environment. \vspace{2em}\\
\begin{verbatim}
export X=42
echo $X
unset X
\end{verbatim}
\end{enumerate}
